% 1 引言
% (阐述课题研发的详细背景、应用场景、问题、可能的解决的思路与技术概述等….)
\section{引言}

\subsection{历史背景}
早期的大数据分析主要以离线处理为主,以Hadoop为代表的技术栈很好的解决了大规模数据的分析问题。
然而数据处理的时效性不足, 很难满足高实时需求的场景。
以Storm为代表的流式计算引擎的出现则很好的解决了数据实时处理的问题,提高了数据处理的时效性。
然而,Storm 本身不提供状态管理的能力, 对于聚合等有状态的计算显得无能为力。
Flink 的出现很好的弥补了这一短板,通过引入状态管理以及 Checkpoint 机制,实现了高效的有状态流计算能力。

随着数据实时处理场景的丰富,尤其是在实时数仓场景下,实时关系运算(即 Stream Join) 越来越多的成为数据实时化的难点。
Flink 虽然具备优秀的状态管理能和出色的性能,然而在处理 Join 运算,尤其是3度以上Join时, 性能瓶颈越来越明显。
由于需要在Join两端存放各个输入的数据状态,当Join变多时,状态的数据量急剧扩大,性能也变的难以接受。
产生这个问题的本质原因是Flink等流计算系统以表作为数据模型,而表模型本身是一个二维结构,不包含关系的定义和关系的存储, 在处理关系运算时只能通过Join运算方式实现,成本很高。

\subsection{应用场景}
在蚂蚁集团的大数据应用场景中,尤其是金融风控、实时数仓等场景下,存在大量Join运算,如何提高Join
的时效性和性能成为重要挑战,为此引入了图模型。图模型是一种以点边结构描述实体关系的数据模型,在图模型里面,点代表实体,
边代表关系,数据存储层面点边存放在一起。因此,图模型天然定义了数据的关系同时存储层面物化了点边关系。
基于图模型,蚂蚁集团实现了新一代实时计算 引擎GeaFlow,很好的解决了复杂关系运算实时化的问题。目前GeaFlow已广泛应用于数仓加速、金融风控、知识图谱以及社交网络等场景。

图模式匹配是一种在图结构中寻找与给定模式相同或相似子图的技术,是图计算领域中一类非常经典的算法问题,在金融风控、社交推荐、计算机视觉、生物信息学等领域有着广泛的应用场景。
随着图数据规模的增大,图模式匹配的计算和存储成本急剧上升,基于分布式图计算引擎提供的迭代计算框架,是实现大规模图数据上图模式匹配算法的一种常用手段。

\subsection{问题概述}
本文基于蚂蚁开源的分布式实时图计算引擎TuGraph Analytics提供的⾼阶API编程接口,
在LDBC FinBench测试数据集上,完成指定的图模式匹配算法的实现,并尽可能提升算法的整体性能(包含构图、匹配、输出全部过程)。

\subsection{解决思路概述}
\begin{enumerate}
  \item Input: 图构建;
  \item Processing: 图匹配;
  \item Output: 处理中间结果并得到输出.
\end{enumerate}
