\subsection{资金快进快出问题}
具体问题描述可见 \ref{pro2} 资金快进快出问题一节.

此问题涉及的数据文件有:
\begin{itemize}
  \item Account.csv(点)
  \item AccountTransferAccount.csv(边)
\end{itemize}

\subsubsection{输入}
\begin{center}
\begin{minted}[xleftmargin=5mm]{java}
// 构建点
PWindowSource<IVertex<String, String>> vertices = pipelineTaskCxt
  .buildSource(
    new FileSource<IVertex<String, String>>(
      "Account.csv",
      line -> {
        String[] fields = line.split("\\|");
        // <vertexKey, vertexValue>: <String, String>
        // 此处以 Account 的 Id 作为点的 Id,
        IVertex<String, String> vertex = new ValueVertex<String, String>(
          fields[0], "");
        return Collections.singletonList(vertex);
    }),
    AllWindow.getInstance())
  .withParallelism(sourceParallelism);

// 构建边
PWindowSource<IEdge<String, Double>> edges = pipelineTaskCxt
  .buildSource(
    new FileSource<IEdge<String, Double>>(
      "AccountTransferAccount.csv",
      line -> {
        String[] fields = line.split("\\|");
        // <vertexKey, edgeValue>: <String, Double>
        // 此处以 Account 的 Id 作为点的 Id
        IEdge<String, Double> edge = new ValueEdge<String, Double>(
          fields[0], fields[1], Double.parseDouble(fields[2]));
        // (srcId, targetId, edgeValue)
        return Collections.singletonList(edge);
      }),
      AllWindow.getInstance())
  .withParallelism(sourceParallelism);

// 构造图视图(用于将构造出的点和边关联起来)
GraphViewDesc graphViewDesc = GraphViewBuilder
  .createGraphView(GraphViewBuilder.DEFAULT_GRAPH)
  .withShardNum(iterationParallelism)
  .withBackend(BackendType.Memory)
  .build();
\end{minted}
\end{center}

\subsubsection{核心算法}

此问题中涉及到的图记作 $ G $, 对一个点 $ v \in V(G) $,
\begin{enumerate}
  \item 找到 $ v $ 的所有入边 $ wv \in E(G), w \in V(G) $(至少一条),
    对所有入边的值求和 $ M=\sum_{i=1}^{m} val(w_iv) $, 其中的 $ m $ 是 $ v $
    的入边的数量, $ val(w_iv) $ 是 $ v $ 的入边 $ w_iv $ 的值;
  \item 找到 $ v $ 的所有出边 $ vx \in E(G), x \in V(G) $(至少一条),
    对所有出边的值求和 $ N=\sum_{j=1}^{n} val(vx_j)$, 其中 $ n $ 是 $ v $
    的出边的数量, $ val(vx_i) $ 是 $ v $ 的出边 $ vx_i $ 的值;
  \item 得到结果 $ M / N $.
\end{enumerate}

\begin{center}
\begin{minted}[xleftmargin=5mm]{java}
public void compute(
  String vertexId, Iterator<Double> messageIterator) {
  List<IEdge<String, Double>> edges = this.context.edges().getOutEdges();

  // 第一轮迭代向所有出边的目标点发送边值
  if (this.context.getIterationId() == 1L) {
    for (IEdge<String, Double> edge : edges) {
      this.context.sendMessage(edge.getTargetId(), edge.getValue());
    }
  } else {
    // 第二轮迭代接收发来的信息
    double in_sum = 0.0, out_sum = 0.0;

    while (messageIterator.hasNext()) {
      Double msg = messageIterator.next();
      in_sum += msg;
    }
    if (in_sum == 0.0)
      return;

    for (IEdge<String, Double> edge : edges) {
      out_sum += edge.getValue();
    }
    if (out_sum != 0.0) {
      double res = in_sum / out_sum;
      String resStr = "";
      resStr = String.format("%.2f", res);
      this.context.setNewVertexValue(resStr);
    }
  }
}
\end{minted}
\end{center}

\subsubsection{输出}
\begin{center}
\begin{minted}[xleftmargin=5mm]{java}
PWindowStream<IVertex<String, String>> result = pipelineTaskCxt
  .buildWindowStreamGraph(vertices, edges, graphViewDesc)
  .compute(new Case3Algorithms(2)) // 进行 2 轮迭代
  .compute(iterationParallelism)
  .getVertices();

SinkFunction<String> sink = ExampleSinkFunctionFactory.getSinkFunction(conf);
result
  .filter(v -> !v.getValue().equals("")) // // 过滤出符合要求的点
  .map(v -> String.format("%s|%s", v.getId(), v.getValue()))
  .withParallelism(mapParallelism)
  .sink(sink) // 写入文件
  .withParallelism(sinkParallelism);
\end{minted}
\end{center}
