% 实践总结
% (描述所完成的任务结果情况,以及可能的特色与应用意义、价值等,以及通过课程学习
% 和实践,一些其他方面的收获或成果等….)
\section{实践总结}
在这次 CCF-BDCI 竞赛中,我们参与了基于 TuGraph Analytics 的高性能图模式匹配算法设计的大数据项目。这个项目是基于蚂蚁集团开源的分布式实时图计算引擎 TuGraph Analytics。在现代大数据应用场景中,图模型被引入作为一种高效地处理大量 Join 操作的数据模型。我们的主要任务是在测试数据集上利用分布式图计算引擎提供的迭代计算框架来实现大规模图数据上的图模式匹配算法,并尽可能提升这些算法的整体性能。我们共同提交了对这四个问题的解决方案,尽管有些问题的解决方案最终未能够通过官方数据集的评测,但仍然从中学习到了许多宝贵的经验。

\subsection{交易闭环搜索问题}
我们的目标是找出图中所有形成闭环的账户。这个问题的核心算法是寻找图中三角形$ K^3 $的个数。
我们首先通过构建点和边来创建图视图,然后实施了一种消息传递算法。
每个点 $ A $ 会向所有相邻点发送消息,并且当这个消息在图中传递时,消息的内容会根据边的连接方式发生变化,计算消息三次传递后仍然返回原点的闭环数量,顺利通过了官方数据集评测。

这种方法让我们深入理解了图计算的核心思想和消息传递机制。我们学会了如何有效地在图结构中传递信息,并且学会使用迭代器来处理图结构中的信息传递机制,对我们解决以下几道题目起到了巨大的帮助。

\subsection{资金快进快出问题}
这个问题的目标是找出所有账户的资金流入流出比。我们主要需要对图中每个点的入边和出边的处理。
我们首先构建了点和边来形成图视图,然后使用迭代计算来找出每个点的资金流入流出比。
顺利通过了这道题目的评测。

在这个过程中,我们学会了如何处理和分析涉及大量交易的图数据。特别是,我们了解到了如何在图模型中实现复杂的计算,这对于理解现实世界中的金融交易网络非常有帮助。

\subsection{担保金额汇总问题}
\textbf{初始算法}:我们的目标是汇总所有符合特定条件的人的贷款金额。我们的初始算法包括两个主要步骤:首先连接 Person.csv 和 Loan.csv 文件,生成一个包含个人ID和此人申请的贷款总额的新表。接着,我们使用这个新表和 PersonGuaranteePerson 表来分析满足条件的人的信息。我们使用图迭代和消息传递算法来实现这一过程,提交后未能通过官方评测但我们考虑到数据集中可能存在环形情况而我们算法无法解决这个问题因此我们改进了算法。

\textbf{改进后的算法}:在反思我们的初始方法后,我们改进了第二步的计算方法,通过封装点为对象,并将传递的数据类型改为字符串类型,以便更好地控制信息流并去重。我们通过四轮迭代的方法来累加所有符合条件的人的贷款金额。尽管我们的改进在自己选择的测试集中表现良好,但最终仍未能通过官方的评测。

尽管我们没有解决这个问题,但我们从中学到了如何处理图中的环形结构,以及如何优化图计算算法以提高性能和准确性。
我们还学到了在面对复杂问题时如何进行迭代改进和创新的重要性。

\subsection{个人贷款统计问题}
\textbf{图构建和数据准备}:我们首先创建了一个基于图的数据结构,这涉及将多个数据源整合为图中的点和边。为了保证数据的准确对应和防止ID冲突,我们采用了特定的策略来区分不同类型的点。

\textbf{迭代计算过程}:我们设计的算法核心是通过多轮迭代计算来传递和聚合数据。在每一轮迭代中,点会根据其连接的边,传递或接收特定的数据值。这个过程涉及了对每个点的值进行更新和累加,以此来实现从一级点到下一级点的数据传递。

\textbf{结果生成与优化}:最终,我们通过过滤和汇总操作生成所需的结果。虽然我们的算法在自我测试中表现良好,可以通过我们设置的数据集,但在官方评测中未能成功,或许在处理边界情况和特殊场景时,我们仍有未考虑到的情况,但由于时间以及能力有限,我们没有再对算法进行优化改进。
